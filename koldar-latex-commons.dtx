%\iffalse meta-comment
% This dtx file will be evaluated twice: 
% - the frist time with comments in place, 
% - the second time where the first percentage symbol is removed.
% 
% In the second compilation, to avoid computing the content of a comment, ensure it is wrapped in iffalse and fi (like here).
% Thus:
% - code in comments wrapped by iffalse are ignored in both the first and in the second compilation;
% - code in comments without iffalse are ignored in the first compilation BUT not in the second compilation;
% - code outside comments are evaluated in both the first compilation AND in the second compilation;
%
% Hence the rule is:
% - code outside comments is the library actual code;
% - code inside comments is latex code that will be put in the library documentation;
% - code inside comments wrapped in iffalse are actual comments that are ignored by latex (in both compilations)
%
% lines within <*documentation> and </documentation> are put into .sty verbatim
%
% For other information, see https://ctan.mirror.garr.it/mirrors/ctan/info/dtxtut/dtxtut.pdf
%\fi

% \iffalse meta-comment
%
% Copyright (C)2021 by Massimo Bono
%
% This file may be distributed and/or modified under the
% conditions of the LaTeX Project Public License, either
% version 1.3 of this license or (at your option) any later
% version.  The latest version of this license is in:
%
%      http://www.latex-project.org/lppl.txt
%
% and version 1.3 or later is part of all distributions of
% LaTeX version 2005/12/01 or later.
%
% \fi

% \iffalse
%<*documentation>
    \NeedsTeXFormat{LaTeX2e}[2005/12/01]
    \ProvidesPackage{koldar-latex-commons}[2021/01/12 v1.0.0 Set of utilities I find useful]
%</documentation>
% \iffalse
% %%%%%%%%%%%%%%%%%%%%%%%%%%%%%%%%%%%%%%%%%%%%%%%%%%%%%%%%%%%%%%%%%%%%%%%%%%%%%%%%%%%%%%%%%%%%
% %%%%%%%%%%%%%%%%%%%%%%%%%%%%%%%%%%%%%%%%%%%%%%%%%%%%%%%%%%%%%%%%%%%%%%%%%%%%%%%%%%%%%%%%%%%%
% %%%%%%%%%%%%%%%%%%% CONTENT COMPILED DURING THE FIRST COMPILATION %%%%%%%%%%%%%%%%%%%%%%%%%%
% %%%%%%%%%%%%%%%%%%%%%%%%%%%%%%%%%%%%%%%%%%%%%%%%%%%%%%%%%%%%%%%%%%%%%%%%%%%%%%%%%%%%%%%%%%%%
% %%%%%%%%%%%%%%%%%%%%%%%%%%%%%%%%%%%%%%%%%%%%%%%%%%%%%%%%%%%%%%%%%%%%%%%%%%%%%%%%%%%%%%%%%%%%
% \fi

%<*driver>

\documentclass{ltxdoc}
\usepackage{koldar-latex-commons}
% begin of Documentation option
\EnableCrossrefs
\CodelineIndex
\RecordChanges
%  End of Documentation option
\begin{document}
    \DocInput{koldar-latex-commons.dtx}
\end{document}

%</driver>

% \fi

% \iffalse
% %%%%%%%%%%%%%%%%%%%%%%%%%%%%%%%%%%%%%%%%%%%%%%%%%%%%%%%%%%%%%%%%%%%%%%%%%%%%%%%%%%%%%%%%%%%%
% %%%%%%%%%%%%%%%%%%%%%%%%%%%%%%%%%%%%%%%%%%%%%%%%%%%%%%%%%%%%%%%%%%%%%%%%%%%%%%%%%%%%%%%%%%%%
% %%%%%%%%%%%%%%%%%%% CONTENT COMPILED DURING THE SECOND COMPILATION %%%%%%%%%%%%%%%%%%%%%%%%%
% %%%%%%%%%%%%%%%%%%%%%%%%%%%%%%%%%%%%%%%%%%%%%%%%%%%%%%%%%%%%%%%%%%%%%%%%%%%%%%%%%%%%%%%%%%%%
% %%%%%%%%%%%%%%%%%%%%%%%%%%%%%%%%%%%%%%%%%%%%%%%%%%%%%%%%%%%%%%%%%%%%%%%%%%%%%%%%%%%%%%%%%%%%
% \fi

% \doccomment{Checksum to ensure the package has been correctly downloaded}
%
% \CheckSum{0}
%
%
% \doccomment{Change log of the library}
%
%\changes{v1.0.0}{2021/01/12}{Initial version}
%
% \doccomment{Fetches filedate, fileversion and fileinfo from ProvidesPackage entry}
%
%\GetFileInfo{koldar-latex-commons.sty}
%
% \begin{documentcomment}
% %%%%%%%%%%%%%%%%%%%%%%%%%%%%%%%%%%%%%%%%%%%%%%%%%%%%%%%%%%%%%%%%%%%%%%%%%%%%%%%%%%%%%%%%%%%%
% %%%%%%%%%%%%%%%%%%%%%%%%%%%%%%%%%%%%%%%%%%%%%%%%%%%%%%%%%%%%%%%%%%%%%%%%%%%%%%%%%%%%%%%%%%%%
% %%%%%%%%%%%%%%% START OF THE LIBRARY AND OF THE LIBRARY DOCUMENTATION %%%%%%%%%%%%%%%%%%%%%%
% %%%%%%%%%%%%%%%%%%%%%%%%%%%%%%%%%%%%%%%%%%%%%%%%%%%%%%%%%%%%%%%%%%%%%%%%%%%%%%%%%%%%%%%%%%%%
% %%%%%%%%%%%%%%%%%%%%%%%%%%%%%%%%%%%%%%%%%%%%%%%%%%%%%%%%%%%%%%%%%%%%%%%%%%%%%%%%%%%%%%%%%%%%
% \end{documentcomment}
%
%
% \title{Koldar-latex-commons}
% \author{Massimo Bono \\ Version \fileversion \\ Released Date \filedate}
% \date{\today}
% \maketitle
% \clearpage
% \tableofcontents
% \clearpage
% \section{Why this library?}
% When writing latex code, I often required the same commands and environment to quickly writing documents and report. I needed simple functions that performed
% simple but common actions. Furthermore, often I needed other developers which were not fully accustumed to writing latex documents to edit the documents themselves.
%
% This library aims to solve these 2 tasks:
% \begin{itemize}
%   \item providing simple but commonly needed commands and environments;
%   \item providing simple alias that performs actions in a way that hide latex complexity to new developers;
% \end{itemize}
%
% \doccomment{
%%%%%%%%%%%%%%%%%%%%%%%%%%%%%%%%%%%%%%%%%%%%%
%% PACKAGES
%%%%%%%%%%%%%%%%%%%%%%%%%%%%%%%%%%%%%%%%%%%%%
% }
% 
% \iffalse STANDARD \fi
% \iffalse
% Since RequirePackage clashes when performing the second latex compilation, we need to exclude the section from the second compilation. Hence we wrap it using the supper \ iffalse
\RequirePackage{xkeyval} 
\RequirePackage{etoolbox}
\RequirePackage{subcaption}
\RequirePackage{float}
\RequirePackage{xparse}
\RequirePackage{xstring}
\RequirePackage{graphicx}
\RequirePackage{diagbox}
\RequirePackage{listings}

% \iffalse ARITHEMTIC \fi

\RequirePackage{calculator}

% \iffalse PROOF READING \fi

\RequirePackage[T1]{fontenc}
\RequirePackage[utf8]{inputenc}
% \iffalse \RequirePackage[english]{babel} \fi

% \iffalse URL WITHIN THE PDF \fi

\RequirePackage{hyperref}

% \iffalse BIBLIOGRAPHY \fi

\RequirePackage[nottoc,numbib]{tocbibind}

% \iffalse TABLES \fi

\RequirePackage{booktabs}
\RequirePackage{makecell}

% \iffalse SYMBOLS \fi

\RequirePackage{pifont}
\RequirePackage{fontawesome}
\RequirePackage{textcomp} % \iffalse TM \fi

% \iffalse URL \fi

\RequirePackage{url}

% \iffalse WARNING NOTES \fi

\RequirePackage{mdframed}

% \iffalse TIKZ \fi

\RequirePackage{tikz}
% \iffalse \RequirePackage{arrows.meta, matrix, arrows,automata,positioning} \fi

% \iffalse FONTS AND COLORS \fi

\RequirePackage{xcolor}
\RequirePackage{cancel}

% \iffalse THEOREMS & MATHS \fi

\RequirePackage{amsmath}
\RequirePackage{amsthm}
\RequirePackage{amssymb}
\RequirePackage{mathtools}

%\iffalse ALGORITHMS \fi

\RequirePackage[ruled,vlined,linesnumbered]{algorithm2e}

% \iffalse GLOSSARIES \fi

\RequirePackage{glossaries}
\RequirePackage{glossaries-extra}

% \iffalse
%%%%%%%%%%%%%%%%%%%%%%%%%%%%%%%%%%%%%%%%%%%%%
%% PACKAGES OPTIONS OF CUSTOM COMMANDS
%%%%%%%%%%%%%%%%%%%%%%%%%%%%%%%%%%%%%%%%%%%%%
% \fi

% \iffalse
%%%%%%%%%%%%%%%%%%%%%%%%%%%%%%%%%%%%%%%%%%%%%
%% PROCESS OPTIONS
%%%%%%%%%%%%%%%%%%%%%%%%%%%%%%%%%%%%%%%%%%%%%
% \fi

\ProcessOptions*\relax
% \fi

% \section{Commands provided by the package}
% We categorize the provided commands and environment according to what they achieve.

% \subsection{Utils}
\makeatletter

% \begin{macro}{\doccomment}
% When writing a dtx, use this command \textbf{within} a comment to write a comment that is ignored both in the first and in the second docstrip compilation
%
%
%
% \marg{text} text to put in the comment
%    \begin{macrocode}
    \ProvideDocumentCommand{\doccomment}{m}{%
        \iffalse {#1} \fi%
    }%
%    \end{macrocode}
% \end{macro}


\makeatother


\makeatletter

% \begin{macro}{\todo}
% Adds a todo embedded in the text (colored blue)
% \oarg{star} if a star is present, we will put the todo as a footnote
% \marg{text} text to put in the todo
%    \begin{macrocode}
\ProvideDocumentCommand{\todo}{s m}{%
    \IfBooleanTF{#1}{%
        \footnote{\color{blue} #2}%
    }{%
        {\color{blue} #2}%
    }%
}
%    \end{macrocode}
% \end{macro}


% \begin{macro}{\code}
% Writes a text in a fixed font. For example \verb|\code{foo}| generates \code{foo}.
% \marg{text} text to print in a formatted way
%    \begin{macrocode}
\ProvideDocumentCommand{\code}{m}{%
    \texttt{#1}%
}
%    \end{macrocode}
% \end{macro}

% \begin{macro}{\isInMath}
% If we are in math mode, do something; otherwise do something else
% \marg{text1} section to put in the latex if we are in math mode;
% \marg{text2} section to put in the latex if we are not in math mode;
%    \begin{macrocode}
\ProvideDocumentCommand{\isInMath}{m m}{%
    \ifmmode{#1}\else{#2}\fi%
}
%    \end{macrocode}
% \end{macro}

% \begin{macro}{\wrapMath}
% Write the text wrapped in math environment. If we are already in math mode, the command does nothing
% \marg{text} text that needs to be wrapped
%    \begin{macrocode}
\ProvideDocumentCommand{\wrapMath}{m}{%
    \ifmmode{#1}\else{$#1$}\fi%
}
%    \end{macrocode}
% \end{macro}

% \begin{macro}{\dquote}
% Write the text wrapped in single quotes (for instance \verb|\dquote{foo}| generates \dquote{foo})
% \marg{text} text that needs to be wrapped
%    \begin{macrocode}
\ProvideDocumentCommand{\dquote}{m}{%
    ``{#1}''%
}
%    \end{macrocode}
% \end{macro}

% \begin{macro}{\squote}
% Write the text wrapped in single quotes (for instance \verb|\squote{foo}| generates \squote{foo})
% \marg{text} text that needs to be wrapped
%    \begin{macrocode}
\ProvideDocumentCommand{\squote}{m}{%
    \isInMath%
        {\mbox{`}{#1}\mbox{'}}%
        {`{#1}'}%
}
%    \end{macrocode}
% \end{macro}


\makeatother
\makeatletter

% \begin{macro}{\eg}
% Write the for example acronym. (\eg{})
%
%    \begin{macrocode}
\ProvideDocumentCommand{\eg}{}{%
    e.g.,%
}
%    \end{macrocode}
% \end{macro}

% \begin{macro}{\ie}
% Write the for instance acronym. (\ie{})
%
%    \begin{macrocode}
\ProvideDocumentCommand{\ie}{}{%
    i.e.,%
}
%    \end{macrocode}
% \end{macro}

% \begin{macro}{\st}
% Write the \dquote{such that} acronym. (\st{})
%
%    \begin{macrocode}
\ProvideDocumentCommand{\st}{}{%
    s.t.%
}
%    \end{macrocode}
% \end{macro}

% \begin{macro}{\wrt}
% Write the \dquote{with respect to} acronym. (\wrt{})
%
%    \begin{macrocode}
\ProvideDocumentCommand{\wrt}{}{%
    w.r.t.%
}
%    \end{macrocode}
% \end{macro}

% \begin{macro}{\iff}
% Write the if and only if acronym.
%
%    \begin{macrocode}
\ProvideDocumentCommand{\iff}{}{%
    \textit{iff}%
}
%    \end{macrocode}
% \end{macro}


% \begin{macro}{\cite}
% Automatically prepend a space before the citation.
% See https://tex.stackexchange.com/a/369691/145331 for further information
%
% \marg{cite} citation to add. Can also be a comma-separated field
%    \begin{macrocode}
\let\@oldcite\cite
\renewcommand*\cite[1]{~\@oldcite{#1}}
%    \end{macrocode}
% \end{macro}


\makeatother
\makeatletter

% \begin{macro}{\setFontSize}
% Write a text using a font size generated on the fly
% See https://texblog.org/2012/08/29/changing-the-font-size-in-latex/ for further information
%
% \marg{height} width of the letters, in pixel;
%
% \oarg{width} baseline of the letter, in pixel. If missing, it will be $1.2 \cdot height$;
%
% \marg{text} text to write using this font
%    \begin{macrocode}
\ProvideDocumentCommand{\setFontSize}{m o m}{%
    \IfNoValueF{#2}{%
        \fontsize{#1}{#2}\selectfont#3%
    }{%
        
        \MULTIPLY{#1}{1.2}{\setFont@baseline}%
        \fontsize{#1}{\setFont@baseline}\selectfont#3%
    }%
}
%    \end{macrocode}
% \end{macro}

\makeatother
\makeatletter

% \begin{macro}{\paircs}
% print a computer science ordered pair. For example the ordered pair 5,4 can be printed as in \paircs{5}{4}
%
% \marg{item1} first element of the pair
%
% \marg{item2} second element of the pair
%
%    \begin{macrocode}
\ProvideDocumentCommand{\paircs}{m m}{%
    \wrapMath{\langle {#1}, {#2} \rangle }%
}
%    \end{macrocode}
% \end{macro}

% \begin{macro}{\bigO}
% Write the text within a big O notation (for example \verb|$\bigO{n^2}$| generates $\bigO{n^2}$)
%
% \marg{text} the text to diasplay in the big O notation
%    \begin{macrocode}
\ProvideDocumentCommand{\bigO}{m}{%
    \wrapMath{\mathcal{O}(#1)}%
}
%    \end{macrocode}
% \end{macro}

% \begin{macro}{\stacksymbols}
% Stack 2 symbols on on top of the other. For instance \verb|$\stacksymbols{lim}{x \rightarrow 5}$| yields $\stacksymbols{lim}{x \rightarrow 5}$
%
% \marg{top} item to put on top
%
% \marg{bottom} item to put on bottom
%    \begin{macrocode}
\ProvideDocumentCommand{\stacksymbols}{m m}{%
    \wrapMath{\stackrel{\mathclap{#1}}{#2}}%
}
%    \end{macrocode}
% \end{macro}

% \begin{macro}{\NPComplete}
% Displays \NPComplete{}
%    \begin{macrocode}
\ProvideDocumentCommand{\NPComplete}{}{%
    \code{NP}-Complete%
}
%    \end{macrocode}
% \end{macro}

% \begin{macro}{\NPHard}
% Displays \NPHard{}
%    \begin{macrocode}
\ProvideDocumentCommand{\NPHard}{}{%
    \code{NP}-Hard%
}
%    \end{macrocode}
% \end{macro}

% \begin{macro}{\NPHardness}
% Displays \NPHardness{}
%    \begin{macrocode}
\ProvideDocumentCommand{\NPHardness}{}{%
    \code{NP}-Hardness%
}
%    \end{macrocode}
% \end{macro}

\makeatother
\makeatletter

\ProvideDocumentCommand{\true}{}{%
    \code{true}%
}

\ProvideDocumentCommand{\false}{}{%
    \code{false}%
}

\ProvideDocumentCommand{\nil}{}{%
    \texttt{NIL}%
}

\makeatother
\makeatletter

% \begin{macro}{\lstref}
%  Add a reference to a listing created using \verb|\begin{lstlisting}|
%
% \marg{item1} the label of the section
%
%    \begin{macrocode}
\ProvideDocumentCommand{\lstref}{m}{%
    Listing \ref{#1}%
}
%    \end{macrocode}
% \end{macro}

% \begin{macro}{\listingref}
%  Add a reference to a listing created using \verb|\begin{lstlisting}|
%
% \marg{item1} the label of the section
%
%    \begin{macrocode}
\ProvideDocumentCommand{\listingref}{m}{%
    Listing \ref{#1}%
}
%    \end{macrocode}
% \end{macro}

% \begin{macro}{\subsubsectionref}
%  Add a reference to a \verb|\subsubsection|
%
% \marg{item1} the label of the subsubsection
%
%    \begin{macrocode}
\ProvideDocumentCommand{\subsubsectionref}{m}{%
    Sub Sub Section \ref{#1}%
}
%    \end{macrocode}
% \end{macro}

% \begin{macro}{\sectionref}
%  Add a reference to a \verb|\section|
%
% \marg{item1} the label of the section
%
%    \begin{macrocode}
\ProvideDocumentCommand{\sectionref}{m}{%
    Section \ref{#1}%
}
%    \end{macrocode}
% \end{macro}

% \begin{macro}{\equationref}
%  Add a reference to an equation
%
% \marg{item1} the label of the equation
%
%    \begin{macrocode}
\ProvideDocumentCommand{\equationref}{m}{%
    Equation \eqref{#1}%
}
%    \end{macrocode}
% \end{macro}

% \begin{macro}{\alglineref}
%  Add a reference to a line within an algorithm
%
% \marg{item1} the label of the line
%
%    \begin{macrocode}
\ProvideDocumentCommand{\alglineref}{m}{%
    Line \ref{#1}%
}
%    \end{macrocode}
% \end{macro}

% \begin{macro}{\alglinesref}
% Add a reference to a range of lines within an algorithm.
%
% \marg{first} first line of the range;
%
% \marg{last} last line of the range;
%
%    \begin{macrocode}
\ProvideDocumentCommand{\alglinesref}{m m}{%
    Lines \ref{#1}--\ref{#2}%
}
%    \end{macrocode}
% \end{macro}

% \begin{macro}{\algref}
% Add a reference to an algorithm
%
% \marg{ref} the label of the algorithm
%
%    \begin{macrocode}
\ProvideDocumentCommand{\algref}{m}{%
    Algorithm \ref{#1}%
}
%    \end{macrocode}
% \end{macro}

% \begin{macro}{\defref}
% Add a reference to a definition
%
% \marg{ref} the label of the definition
%
%    \begin{macrocode}
\ProvideDocumentCommand{\defref}{m}{%
    Definition \ref{#1}%
}
%    \end{macrocode}
% \end{macro}

% \begin{macro}{\defsref}
% Add a reference to a range of definitions
%
% \marg{start} the label of the first definition
%
% \marg{end} the label of the second definition
%
%    \begin{macrocode}
\ProvideDocumentCommand{\defsref}{m m}{%
    Definitions \ref{#1}--\ref{#2}%
}
%    \end{macrocode}
% \end{macro}

% \begin{macro}{\figref}
% Add a reference to a figure
%
% \marg{ref} the label of the figure
%
%    \begin{macrocode}
\ProvideDocumentCommand{\figref}{m}{%
    Figure \ref{#1}%
}
%    \end{macrocode}
% \end{macro}

% \begin{macro}{\figsref}
% Add a reference to a range of figure
%
% \marg{start} the label of the first figure in the range (inclusive)
%
% \marg{end} the label of the last figure in the range (inclusive)
%
%    \begin{macrocode}
\ProvideDocumentCommand{\figsref}{m m}{%
    Figures \ref{#1}--\ref{#2}%
}
%    \end{macrocode}
% \end{macro}

\ProvideDocumentCommand{\r@figrefs}{m}{%
    \@ifnextchar\bgroup{, \ref{#1}\r@figrefs}{ and \ref{#1}}%
}

% \begin{macro}{\figrefs}
% Add a reference to a list fo figures. The figures are joined via \verb|,|,
% except the last one since it is concatenated using the word \verb|and|.
% For example \verb|\figrefs{fig:01}{fig:02}{fig:03}|
%
% \marg{...} variadic elements. 
%
%    \begin{macrocode}
\ProvideDocumentCommand{\figrefs}{m}{%
    Figures \ref{#1}\r@figrefs%
}
%    \end{macrocode}
% \end{macro}

% \begin{macro}{\tblref}
% Add a reference to a table
%
% \marg{ref} the label of the table
%
%    \begin{macrocode}
\ProvideDocumentCommand{\tblref}{m}{%
    Table \ref{#1}%
}
%    \end{macrocode}
% \end{macro}

% \begin{macro}{\thmref}
% Add a reference to a theorem
%
% \marg{ref} the label of the theorem
%
%    \begin{macrocode}
\ProvideDocumentCommand{\thmref}{m}{%
    Theorem \ref{#1}%
}
%    \end{macrocode}
% \end{macro}

% \begin{macro}{\lemmaref}
% Add a reference to a lemma
%
% \marg{ref} the label of the lemma
%
%    \begin{macrocode}
\ProvideDocumentCommand{\lemmaref}{m}{%
    Lemma \ref{#1}%
}
%    \end{macrocode}
% \end{macro}

% \begin{macro}{\exampleref}
% Add a reference to an example
%
% \marg{ref} the label of the example
%
%    \begin{macrocode}
\ProvideDocumentCommand{\exampleref}{m}{%
    Example \ref{#1}%
}
%    \end{macrocode}
% \end{macro}

\makeatother
\makeatletter

\ProvideDocumentCommand{\doublePlus}{}{%
    \ifmmode{+\!\!+}\else{$+\!\!+$}\fi%
}



% draw a square.
% param1: fill color red!50
% param2: border color (default to black)
\ProvideDocumentCommand{\drawFilledSquare}{m O{black}}{%
    \begin{tikzpicture}%
        \node [rectangle,draw={#2},fill={#1}] (m) at (0,0) {};%
    \end{tikzpicture}%
}

% draw a "v" representing a checkbox which has been checked
\ProvideDocumentCommand{\checked}{}{%
\tikz\fill[scale=0.4](0,.35) -- (.25,0) -- (1,.7) -- (.25,.15) -- cycle;%
}

% draw a "x" representing a checkbox which has been checked
\ProvideDocumentCommand{\unchecked}{}{%
\tikz\fill[scale=0.4]%
    (-0.35,+0.35) -- (+0.00,+0.07) --%
    (+0.40,+0.40) -- (+0.07,+0.00) --%
    (+0.35,-0.35) -- (+0.00,-0.07) --%
    (-0.40,-0.40) -- (-0.07,+0.00) --%
    cycle;%
}

\makeatother
% Commands which are just alias for common latex operations

%%%%%%%%%%%%%%%%%%%%%%%%%%%%%%%%%%%%%%%%%%%%%
% WARNING AND NOTES
%%%%%%%%%%%%%%%%%%%%%%%%%%%%%%%%%%%%%%%%%%%%%

%\begin{environment}{attention}
% Show an attention box. The body of the environment is the body of the box 
%
%
%    \begin{macrocode}
\ProvideDocumentEnvironment{attention}{}{%
    \par%
        \begin{mdframed}[linewidth=3pt,linecolor=red]%
            \begin{list}{}{\leftmargin=1cm \labelwidth=\leftmargin}%
                \item[\Large\faBomb]%
}{%
            \end{list}%
        \end{mdframed}%
    \par
}
%    \end{macrocode}
% \end{environment}

%\begin{environment}{warning}
% Show a warning box. The body of the environment is the body of the box
%
%    \begin{macrocode}
\ProvideDocumentEnvironment{warning}{}{%
    \par%
        \begin{mdframed}[linewidth=3pt,linecolor=orange]%
            \begin{list}{}{\leftmargin=1cm \labelwidth=\leftmargin}%
                \item[\Large\faWarning]%
}{%
            \end{list}%
        \end{mdframed}%
    \par
}
%    \end{macrocode}
% \end{environment}

%\begin{environment}{info}
% Show an note box. The body of the environment is the body of the box
%
%    \begin{macrocode}
\ProvideDocumentEnvironment{info}{}{%
    \par%
        \begin{mdframed}[linewidth=3pt,linecolor=blue]%
            \begin{list}{}{\leftmargin=1cm \labelwidth=\leftmargin}%
                \item[\Large\faBook]%
}{%
            \end{list}%
        \end{mdframed}%
    \par
}
%    \end{macrocode}
% \end{environment}

%\begin{environment}{note}
% Alias of "info"
%
%    \begin{macrocode}
\ProvideDocumentEnvironment{note}{}{%
    \par%
        \begin{mdframed}[linewidth=3pt,linecolor=blue]%
            \begin{list}{}{\leftmargin=1cm \labelwidth=\leftmargin}%
                \item[\Large\faBook]%
}{%
            \end{list}%
        \end{mdframed}%
    \par
}
%    \end{macrocode}
% \end{environment}

%%%%%%%%%%%%%%%%%%%%%%%%%%%%%%%%%%%%%%%%%
% FLOATS
%%%%%%%%%%%%%%%%%%%%%%%%%%%%%%%%%%%%%%%%%

%\begin{environment}{horizontalpage}
% Setup the a horizontal page instead of the classic portait page
%
%    \begin{macrocode}
\ProvideDocumentEnvironment{horizontalpage}{+b}{%
    \begin{sidewaysfigure}%
        #1%
    \end{sidewaysfigure}%
}{%
}
%    \end{macrocode}
% \end{environment}

%%%%%%%%%%%%%%%%%%%%%%%%%%%%%%%%%%%%%%%%%
% TEX COMMONS
%%%%%%%%%%%%%%%%%%%%%%%%%%%%%%%%%%%%%%%%%

% \begin{macro}{\italic}
% Write something in italic. Alias of \verb|\textit|
%
% \marg{ref} the text to write in italic
%
%    \begin{macrocode}
\ProvideDocumentCommand{\italic}{m}{%
    \textit{#1}%
}
%    \end{macrocode}
% \end{macro}

%%%%%%%%%%%%%%%%%%%%%%%%%%%%%%%%%%%%%%%%%
% GLOSSARIES
%%%%%%%%%%%%%%%%%%%%%%%%%%%%%%%%%%%%%%%%%

% \begin{macro}{\acronymRef}
% Display the given acronym 
%
% \marg{ref} name of the acronym to print
%
%    \begin{macrocode}
\ProvideDocumentCommand{\acronymRef}{m}{%
   \gls{#1}%
}
%    \end{macrocode}
% \end{macro}


% \begin{macro}{\acrref}
% Display the given acronym 
%
% \marg{ref} name of the acronym to print
%
%    \begin{macrocode}
\ProvideDocumentCommand{\acrref}{m}{%
   \gls{#1}%
}
%    \end{macrocode}
% \end{macro}

% \begin{macro}{\acref}
% Display the given acronym 
%
% \marg{acronym name} name of the acronym to print
%
%    \begin{macrocode}
\ProvideDocumentCommand{\acref}{m}{%
   \gls{#1}%
}
%    \end{macrocode}
% \end{macro}

% \begin{macro}{\glossaryRef}
% Display a glossary entry with the first letter as miniscule.
%This is just a more rememberable alias of \verb|\gls|
%
% \marg{glossary entry} name of the glossary entry to cite. 
%
%    \begin{macrocode}
\ProvideDocumentCommand{\glossaryRef}{m}{%
    \gls{#1}%
}
%    \end{macrocode}
% \end{macro}

% \begin{macro}{\GlossaryRef}
% Display a glossary entry with the first letter as capital.
%This is just a more rememberable alias of \verb|\Gls|
%
% \marg{glossary entry} name of the glossary entry to cite. 
%
%    \begin{macrocode}
\ProvideDocumentCommand{\GlossaryRef}{m}{%
    \Gls{#1}%
}
%    \end{macrocode}
% \end{macro}

% \begin{macro}{\glsref}
% Display a glossary entry with the first letter as miniscule.
%This is just a more rememberable alias of \verb|\gls|.
%
% \marg{glossary entry} name of the glossary entry to cite. 
%
%    \begin{macrocode}
\ProvideDocumentCommand{\glsref}{m}{%
    \gls{#1}%
}
%    \end{macrocode}
% \end{macro}

% \begin{macro}{\Glsref}
% Display a glossary entry with the first letter as capital.
% This is just a more rememberable alias of \verb|\Gls|.
%
% \marg{glosssary entry} name of the glossary entry to cite. 
%
%    \begin{macrocode}
\ProvideDocumentCommand{\Glsref}{m}{%
    \Gls{#1}%
}
%    \end{macrocode}
% \end{macro}


% % define default listing settings

\lstset{% %
    postbreak=\mbox{\textcolor{red}{$\hookrightarrow$}\space},% %
    frame={single},% %
    breaklines={true},% %
    captionpos={b}% %
}

% %%%%%%%%%%%%%%%%%%%%%%%%%%%%%%%%%%%%%%%%%
% % XML
% %%%%%%%%%%%%%%%%%%%%%%%%%%%%%%%%%%%%%%%%%

\definecolor{xmlgray}{rgb}{0.4,0.4,0.4}
\definecolor{xmldarkblue}{rgb}{0.0,0.0,0.6}
\definecolor{xmlcyan}{rgb}{0.0,0.6,0.6}


\lstdefinelanguage{xml}{
  basicstyle=\footnotesize\ttfamily,
  columns=fullflexible,
  showstringspaces=false,
  commentstyle=\color{xmlgray}\upshape,
  morestring=[b]",
  morestring=[s]{>}{<},
  morecomment=[s]{<?}{?>},
  stringstyle=\color{black},
  identifierstyle=\color{xmldarkblue},
  keywordstyle=\color{xmlcyan},
  morekeywords={xmlns,version,type}
}

% %%%%%%%%%%%%%%%%%%%%%%%%%%%%%%%%%%%%%%%%%
% % JSON
% %%%%%%%%%%%%%%%%%%%%%%%%%%%%%%%%%%%%%%%%%

% % define listing for json language
% % see https://tex.stackexchange.com/a/433961/145331
\definecolor{eclipseStrings}{RGB}{42,0.0,255}
\definecolor{eclipseKeywords}{RGB}{127,0,85}
\colorlet{numb}{magenta!60!black}

\lstdefinelanguage{json}{
    basicstyle=\footnotesize\ttfamily,
    commentstyle=\color{eclipseStrings}, % % style of comment
    stringstyle=\color{eclipseKeywords}, % % style of strings
    numbers=left,
    numberstyle=\scriptsize,
    stepnumber=1,
    numbersep=8pt,
    showstringspaces=false,
    breaklines=true,
    frame={single},
    % %backgroundcolor=\color{gray}, %only if you like
    string=[s]{"}{"},
    comment=[l]{:\ "},
    morecomment=[l]{:"},
    literate=
        *{0}{{{\color{numb}0}}}{1}
         {1}{{{\color{numb}1}}}{1}
         {2}{{{\color{numb}2}}}{1}
         {3}{{{\color{numb}3}}}{1}
         {4}{{{\color{numb}4}}}{1}
         {5}{{{\color{numb}5}}}{1}
         {6}{{{\color{numb}6}}}{1}
         {7}{{{\color{numb}7}}}{1}
         {8}{{{\color{numb}8}}}{1}
         {9}{{{\color{numb}9}}}{1}
}
\makeatletter

% % \theoremstyle{definition}
% % \newtheorem{example}{Example}[section]

\makeatother

% \Finale
\endinput

